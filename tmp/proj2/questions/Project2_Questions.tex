%%%%%%%%%%%%%%%%%%%%%%%%%%%%%%%%%%%%%%%%%%%%%%%%%%%%%%%%%%%%%%%%%%%%%%%%%%%%%%%%%%%%%%%%%%%%%%%%
%
% CSCI 1430 Written Question Template
%
% This is a LaTeX document. LaTeX is a markup language for producing documents.
% Your task is to answer the questions by filling out this document, then to 
% compile this into a PDF document. 
% You will then upload this PDF to `Gradescope' - the grading system that we will use. 
% Instructions for upload will follow soon.
%
% 
% TO COMPILE:
% > pdflatex thisfile.tex
%
% If you do not have LaTeX and need a LaTeX distribution:
% - Departmental machines have one installed.
% - Personal laptops (all common OS): http://www.latex-project.org/get/
%
% If you need help with LaTeX, come to office hours. Or, there is plenty of help online:
% https://en.wikibooks.org/wiki/LaTeX
%
% Good luck!
% James and the 1430 staff
%
%%%%%%%%%%%%%%%%%%%%%%%%%%%%%%%%%%%%%%%%%%%%%%%%%%%%%%%%%%%%%%%%%%%%%%%%%%%%%%%%%%%%%%%%%%%%%%%%
%
% How to include two graphics on the same line:
% 
% \includegraphics[width=0.49\linewidth]{yourgraphic1.png}
% \includegraphics[width=0.49\linewidth]{yourgraphic2.png}
%
% How to include equations:
%
% \begin{equation}
% y = mx+c
% \end{equation}
% 
%%%%%%%%%%%%%%%%%%%%%%%%%%%%%%%%%%%%%%%%%%%%%%%%%%%%%%%%%%%%%%%%%%%%%%%%%%%%%%%%%%%%%%%%%%%%%%%%

\documentclass[11pt]{article}

\usepackage[english]{babel}
\usepackage[utf8]{inputenc}
\usepackage[colorlinks = true,
            linkcolor = blue,
            urlcolor  = blue]{hyperref}
\usepackage[a4paper,margin=1.5in]{geometry}
\usepackage{stackengine,graphicx}
\usepackage{fancyhdr}
\setlength{\headheight}{15pt}
\usepackage{microtype}
\usepackage{times}
% a great python code format: https://github.com/olivierverdier/python-latex-highlighting
\usepackage{pythonhighlight}

\frenchspacing
\setlength{\parindent}{0cm} % Default is 15pt.
\setlength{\parskip}{0.3cm plus1mm minus1mm}

\pagestyle{fancy}
\fancyhf{}
\lhead{Project 2 Questions}
\rhead{CSCI 1430}
\rfoot{\thepage}

\date{}

\title{\vspace{-1cm}Project 2 Questions}


\begin{document}
\maketitle
\vspace{-3cm}
\thispagestyle{fancy}

\section*{Instructions}
\begin{itemize}
  \item One exercise graded for completion only (please include your results, e.g., two-to-three screenshots of your findings). 
  \item Four graded questions.
  \item One ungraded `something to think about'.
  \item Write code where appropriate.
  \item Feel free to include images or equations.
  \item Please make this document anonymous.
  \item On upload, \textbf{Gradescope will ask you to assign question numbers to your pages}. Making each question end with a page break after your answer is a good way to ease this process.
\end{itemize}


\section{Exercise}

\paragraph{E1:} Let's look again at the webcam Fourier decomposition demo which James showed in class. Let's run it in our CSCI 1430 virtual environment \emph{preferably on a computer with a webcam}.

\begin{verbatim}
$ python liveFFT.py
\end{verbatim}

This file contains five parts for you to explore and see the amplitude image, the phase image, and the effect of the reconstructed image.
\begin{itemize}
\item Part 0: Scanning the basis and observing the output image.
\item Part 1: Reconstructions from different numbers of basis frequencies.
\item Part 2: Replacing amplitude and phase with that from a different image.
\item Part 3: Replacing amplitude and phase with that from a noise image.
\item Part 4: Manipulating the amplitude and phase images.
\end{itemize}

Uncomment the different parts and explore the camera feed decomposition! Please include the results of your experimentation, e.g., two-to-three screenshots of what you discover. We'll be grading for completion, not correctness. \emph{Note:} For anonymous grading, try not to put yourself in the camera frame. Show your favourite vector calculus book, wear a mask, use your cat, etc. Extra credit for creative effort.

%%%%%%%%%%%%%%%%%%%%%%%%%%%%%%%%%%%

% Please leave the pagebreak
\pagebreak
\section*{Questions}

\paragraph{Q1:} Imagine we wished to find points in one image which matched to the same world point in another image---so-called feature point correspondence matching. We are tasked with designing an image feature point algorithm which could match world points in the following three pairs of images. 

Please use the included python script \texttt{plot\_corners.py} to find corners using Harris corner detection. Discuss the differences in the returned corners (if any) for each image pair and what real world phenomena or camera effects may have caused these differences. Then discuss which real world phenomena and camera effects might cause us problems when matching these features. Please provide at least one problem per pair.

\emph{RISHLibrary:} \href{RISHLibrary1.jpg}{One} \href{RISHLibrary2.jpg}{Two} | \emph{Chase:} \href{Chase1.jpg}{One} \href{Chase2.jpg}{Two} | \emph{LaddObservatory:} \href{LaddObservatory1.jpg}{One} \href{LaddObservatory2.jpg}{Two}

%%%%%%%%%%%%%%%%%%%%%%%%%%%%%%%%%%%
\paragraph{A1:} Your answer here.



%%%%%%%%%%%%%%%%%%%%%%%%%%%%%%%%%%%

% Please leave the pagebreak
\pagebreak
\paragraph{Q2:} In the Harris corner detector, what do the eigenvalues of the 'M' second moment matrix represent? Discuss both how they relate to image intensity and how we can interpret them geometrically.

%%%%%%%%%%%%%%%%%%%%%%%%%%%%%%%%%%%
\paragraph{A2:} Your answer here.



%%%%%%%%%%%%%%%%%%%%%%%%%%%%%%%%%%%

% Please leave the pagebreak
\pagebreak
\paragraph{Q3:} Given a feature point location, the SIFT algorithm converts a 16$\times$16 patch around the feature point into a 128$\times$1 descriptor of the gradient magnitudes and orientations therein. Write pseudocode \emph{with matrix/array indices} for these steps.

\emph{Notes:} Do this for just one feature point at one scale; ignore the overall feature point orientation; ignore the Gaussian weighting; ignore all normalization post-processing; ignore image boundaries; ignore sub-pixel interpolation and just pick an arbitrary center within the 16$\times$16 for your descriptor. Please just explain in pseudocode how to go from the 16$\times$16 patch to the 128$\times$1 vector. You are free to simplify the gradient computation.

%%%%%%%%%%%%%%%%%%%%%%%%%%%%%%%%%%%
\paragraph{A3:} Your answer here.


% Please leave the pagebreak
\pagebreak
\paragraph{Q3:} Given a feature point location, the SIFT algorithm converts a 16$\times$16 patch around the feature point into a 128$\times$1 descriptor of the gradient magnitudes and orientations therein. Write pseudocode \emph{with matrix/array indices} for these steps.

\emph{Notes:} Do this for just one feature point at one scale; ignore the overall feature point orientation; ignore the Gaussian weighting; ignore all normalization post-processing; ignore image boundaries; ignore sub-pixel interpolation and just pick an arbitrary center within the 16$\times$16 for your descriptor. Please just explain in pseudocode how to go from the 16$\times$16 patch to the 128$\times$1 vector.

%%%%%%%%%%%%%%%%%%%%%%%%%%%%%%%%%%%
\paragraph{A3:} Your answer here.

\begin{python}
# You can assume access to the image, x and y gradients, and their magnitudes/orientations.
image = imread('rara.jpg')
grad_x = filter(image, 'sobelX')
grad_y = filter(image, 'sobelY')
grad_mag = sqrt( grad_x .^2 + grad_y.^2 )
grad_ori = atan2( grad_y, grad_x )

# Takes in a feature point x,y location and returns a descriptor
def SIFTdescriptor(x, y)
    descriptor = zeros(128,1)


    return descriptor
\end{python}


%%%%%%%%%%%%%%%%%%%%%%%%%%%%%%%%%%%



% Please leave the pagebreak
\pagebreak
\paragraph{Q4:} Explain the difference between the Euclidean distance and the cosine similarity metrics between descriptors. What might their geometric interpretations reveal about when each should be used? Given a distance metric, what is a good method for feature descriptor matching and why?

%%%%%%%%%%%%%%%%%%%%%%%%%%%%%%%%%%%
\paragraph{A4:} Your answer here.



%%%%%%%%%%%%%%%%%%%%%%%%%%%%%%%%%%%

\pagebreak
\paragraph{Something to think about:} In designing a feature point matching algorithm, what characteristics might we wish it to have? How might two world points change in appearance across photographs? Consider that we might allow brightness or contrast changes, or texture changes, or lighting changes, or geometric changes in appearance like rotation and translation in three dimensions or camera perspective effects. All exist between some two photographs of real-world points. 

We are faced with a fundamental trade-off between feature point invariance (how much variation in appearance I allow and still say that two points are the same) and discriminative power (our ability to say that two points are different or the same at all). 

How should we design for this trade-off?



% If you really need extra space, uncomment here and use extra pages after the last question.
% Please refer here in your original answer. Thanks!
%\pagebreak
%\paragraph{AX.X Continued:} Your answer continued here.



\end{document}
